%\title{Hebrew document in WriteLatex - מסמך בעברית}
\documentclass{article}
\usepackage[utf8x]{inputenc}
\usepackage[english,hebrew]{babel}
\selectlanguage{hebrew}
\usepackage[top=2cm,bottom=2cm,left=2.5cm,right=2cm]{geometry}


\usepackage{amsfonts,amsmath}

\begin{document}
\selectlanguage{hebrew}
%%%%%%%%%%%%%%%%%%%%%%%%%%%%%%%%%%%%%%%%%%
%%%%%%%% תרגיל 4 %%%%%%%%%%%%%%%%%%%%%%%%%
%%%%%%%%%%%%%%%%%%%%%%%%%%%%%%%%%%%%%%
\section*{תרגיל 4}
נוכיח כי הקבוצות 
$B_1$ 
ו -
$B_2$
, קבוצות קמורות כאשר נתון ש - 
$C \subseteq \mathbb{R}^m $
תחום קמור, ו - 
$A \in \mathbb{R}^{m \times n} $.

%% Set B1
\paragraph*{הקבוצה \L{B_1}}

\begin{equation}
\selectlanguage{english}
B_1 = \{x\in \mathbb{R}^{n}{  } \,|\, x = A^Tz \,,\, z\in C \} \nonumber
\end{equation}

\begin{flushleft}
\selectlanguage{hebrew}
נבדוק אם נקודה שהיא צירוף של שתי נקודות השייכות לקבוצה
$B_1$ 
גם כן שייכת לקבוצה : 
\end{flushleft}

\selectlanguage{english}
\begin{equation}
\begin{align}
\hfilneg &x = A^Tz \,,\, x\in\mathbb{R}^{n} \nonumber \hspace{10000pt minus 1fill} \\
&y = A^Tz  \,,\, y\in\mathbb{R}^{n} \nonumber
\end{align}
\end{equation}

\selectlanguage{hebrew}
\begin{flushleft}
ננדיר נקודה $k$
המקיימת  : 
\end{flushleft}

\begin{equation}
k = \lambda x + (1-\lambda) y \, , \, \lambda \in [0,1] \nonumber
\end{equation}

\begin{equation}
\selectlanguage{english}
k = \lambda x + (1-\lambda) y =\lambda  A^Tz + (1-\lambda) A^Tz = (\lambda+ 1-\lambda)A^Tz = A^Tz \in B_1 \nonumber \hspace{10000pt minus 1fill}
\end{equation}
\selectlanguage{hebrew}
\begin{flushleft}
$B_1$ 
קמורה על פי הגדרה.
\end{flushleft}

%% Set B2
\paragraph*{הקבוצה \L{B_2}}

\begin{equation}
\selectlanguage{english}
B_2 = \{x\in \mathbb{R}^{n}{  } \,|\, Ax  \, \in C \} \nonumber
\end{equation}

\begin{flushleft}
\selectlanguage{hebrew}
נבדוק אם נקודה שהיא צירוף של שתי נקודות השייכות לקבוצה 
$B_2$
גם כן שייכת לקבוצה : 
\end{flushleft}

\begin{equation}
\selectlanguage{english}
\begin{align}
\hfilneg &x : Ax \,,\, \in  C \nonumber \hspace{10000pt minus 1fill} \\
         &y : Ay  \,,\, \in C \nonumber
\end{align}
\end{equation}
\selectlanguage{hebrew}
\begin{flushleft}
ננדיר נקודה $k$
המקיימת  : 
\end{flushleft}

\begin{equation}
k = \lambda x + (1-\lambda) y \, , \, \lambda \in [0,1] \nonumber
\end{equation}

\begin{equation}
\selectlanguage{english}
k = \lambda x + (1-\lambda) y =\lambda  Ax + (1-\lambda) Ay \in B_2 \nonumber \hspace{10000pt minus 1fill}
\end{equation}
\selectlanguage{hebrew}
\begin{flushleft}
סכום של קבוצות קמורות היא קבוצה קמורה ולכן 
$B_2$ 
קמורה על פי הגדרה.
\end{flushleft}

%%%%%%%%%%%%%%%%%%%%%%%%%%%%%%%%%%%%%%%%%%
%%%%%%%% תרגיל 5 %%%%%%%%%%%%%%%%%%%%%%%%%
%%%%%%%%%%%%%%%%%%%%%%%%%%%%%%%%%%%%%%

\pagebreak
\section*{תרגיל 5}
%% סעיף 1 %%
\paragraph*{1. הוכחה }
כי עבור קבוצה $S \in\mathbb{R}^n $
ונקודת צבר $x \in\mathbb{R}^$
, לכל $\epsilon > 0 $ 
הכדור הפתוח $B(x,\epsilon)$
מכיל אינסוף נקודות בקבוצה $S$.

\subparagraph*{נוכיח בשלילה}
\subparagraph*{$(1)$}
נניח כי עבור נקודת צבר $x$ , 
קיימות כמות סופית של נקודות ב - $S$.
\subparagraph*{$(2)$}
עבור נקודת צבר ספציפית $x_a$ .   
הנקודה בעלת סביבה בגודל מינימלי $\epsilon_a$ .      
\subparagraph*{$(3)$}
על פי הגדרה של נקודת צבר, עבור כל $\epsilon_a > 0$  , 
נדרשת שתהיה נקודה בסביבה של $x_a$ , 
הנמצאת ב - $S$.
\subparagraph*{$(4)$}
נבחן את הסביבה של נקודה $x_a$ , 
ברדיוס $\epsilon_a/2$ .
\\
בסביבת הנקודה $x_a$  , 
אין נקודה השייכת ל - $S$. 
)כי אמרנו שהמרחק המינימלי הוא $\epsilon_a$(
\subparagraph*{$(5)$}
כלומר מתקיימת סתירה על פי ההגדרה שעל 
\textbf{כל} $\epsilon>0$
חייבת להיות נקודה השייכת ל - $S$.
\subparagraph*{$(6)$}
על פי הסתירה חייבות להיות נקודות נוספות בסביבה של $x$,
כלומר קיימות אינסוף נקודות .


%% סעיף 2 %%
\paragraph*{2. הוכחה }
שהקבוצה $S$ 
היא קבוצה סגורה אם ורק אם היא מכילה את כל נקודות הצבר שלה.\\
נוכיח את הטענה בשני הכיוונים מכיוון שמדובר בטענת 
\textbf{"אם ורק אם"}
\subparagraph*{
קבוצה $S$ 
היא קבוצה המכילה את כל נקודות הצבר שלה
}
\subparagraph*{$(1)$}
תהי נקודה $x$, 
נקודה במשלים של $(S^c)\,S$ . 
$x$ 
אינה נקודה צבר של $S$, 
)כי התחלנו ש - $S$ 
מכילה את כל נקודות הצבר שלה(.

\subparagraph*{$(2)$}
עבור הקבוצה $S^c$ , 
הנקודה $x$ 
היא נקודת פנים מכיוון שכל הסביבה שלה מוכלת ב - $S^c$.
\subparagraph*{$(3)$}
כלומר, הקבוצה $S^c$,
היא קבוצה פתוחה מכיוון שהיא מכילה את כל נקודות הפנים שלה.
\subparagraph*{$(4)$}
לכן על פי הגדרה, אם $S^c$, 
היא קבוצה פתוחה אז הקבוצה $S$ היא קבוצה סגורה.

\subparagraph*{
 $S$ 
קבוצה סגורה
}
\subparagraph*{$(1)$}
מכיוון ש - $S$, 
היא קבוצה סגורה, אז הקבוצה $S^c$
היא קבוצה פתוחה )כלומר קבוצה המכילה את כל נקודות הפנים שלה(.
\subparagraph*{$(2)$}
נניח שקיימת נקודה $x$, 
שהיא נקודת צבר של $S$.
אז קיימת איזושהי נקודה ב- $S$, 
בסביבה של $x$.
\subparagraph*{$(3)$}
מכיוון שהנקודה $x$
לא שייכת ל - $S$ 
אז היא שייכת ל - $S^c$.
\subparagraph*{$(4)$}
$S^c$ 
היא קבוצה פתוחה, ולכן מכילה את כל נקודות הפנים שלה. כלומר כל נקודה ב - $S^c$
היא נקודת פנים.
\subparagraph*{$(5)$}
נוצרה
\textbf{סתירה} 
כי הנחנו ש - $x$ 
היא נקודת צבר של $S$ 
לכן חייב להיות חיתוך בין הסביבה של $x$ 
ל - $S$

\subparagraph*{
הוכחנו בשני הכיוונים ולכן הוכחנו את הטענה}


%% סעיף 3 %%
\paragraph*{3. הוכחה }
כי אם הסדרה $S$ 
סגורה, והסדרה $x_i \in S$
מתכנסת ל - $x$ : 
$\lim_{i\to\infty} \|x_i-x\| = 0 $ , 
אז $x\in S$

\subparagraph*{}
נוכיח בשלילה כי 
$x\not\in S$.
\subparagraph*{$(1)$}
מכיוון ש - $S$
היא קבוצה סגורה , אז היא מכילה את כל נקודות הצבר שלה. ולכן בהנחת $x\not\in S$ 
נובע ש - $x$
אינה נקודת צבר של $S$
\subparagraph*{$(2)$}
אם $x$ 
לא נקודת צבר של $S$
אז עבור $\epsilon>0$
הכדור הפתוח $B(x,\epsilon)\not\in S$
\subparagraph*{$(3)$}
על פי הנתון, 
$\lim_{i\to\infty} \|x_i-x\| = 0 $ , 
כלומר $x$
צריכה להיות בסביבה של הסדרה $x_i$.
\subparagraph*{$(4)$}
קיבלנו סתירה בין סעיף
$(2)$
ל - 
$(3)$
ולכן $x\in S$



%%%%%%%%%%%%%%%%%%%%%%%%%%%%%%%%%%%%%%%%%%
%%%%%%%% תרגיל 6 %%%%%%%%%%%%%%%%%%%%%%%%%
%%%%%%%%%%%%%%%%%%%%%%%%%%%%%%%%%%%%%%

\pagebreak
\section*{תרגיל 6}
נניח כי קיימים פתרונות נוספים $y_0,z_0$
אשר פותרים את בעיית האופטמיזציה : 
\begin{align}
\selectlanguage{english}
    \|y_0-x_0\|^2 = \|z_0-x_0\|^2 = (min_{y\in C} \| y-y_0\|)^2 = \delta^2
\end{align}
מכיוון ש -$y$
הוא פתרון לבעיית האופטימיזציה, וגם $y\in C$
אז גם נקודה $k_0$
שייכת ל - $C$: 

\begin{equation}
\selectlanguage{english}
    k_0 = \lambda z_0 + (1-\lambda)y_0 \in C \nonumber
\end{equation}
עבור $\lambda = 1/2 $
מתקבל כי הנקודה $k_0$: 

\begin{equation}
\selectlanguage{english}
    k_0 = \lambda z_0 + (1-\lambda)y_0 = \frac{z_0+y_0}{2}\in C \nonumber
\end{equation}
הנקודה $k_0$
היא לא פיתרון של בעיית האופטימיזציה ולכן : 
\begin{align}
&\|k_0-x_0\|^2 \geq \delta^2 \nonumber \\ 
&\|\frac{z_0+y_0}{2}-x_0\|^2 \geq \delta^2  \nonumber \\
&\frac{1}{4}\|z_0+y_0-2x_0\|^2 \geq \delta^2  \nonumber \\
&\frac{1}{4}\|(z_0 - x_0 ) + (y_0 - x_0)\|^2 \geq \delta^2  \nonumber \\
&\frac{1}{4}\left[\|(z_0 - x_0 )\|^2 + 2 \langle z_0 - x_0 , y_0 - x_0 \rangle  + \|(y_0 - x_0)\|^2 \right] \geq \delta^2
\end{align}
נציב במשוואה )2( את משוואה )1( :
\begin{align}
&\frac{1}{4}\left[\delta^2 + 2 \langle z_0 - x_0 , y_0 - x_0 \rangle  + \delta^2 \right] \geq \delta^2 \nonumber \\
&\frac{1}{2}\langle z_0 - x_0 , y_0 - x_0 \rangle \geq \frac{1}{2}\delta^2 \nonumber \\
&\langle z_0 - x_0 , y_0 - x_0 \rangle \geq \delta^2 \nonumber \\
&2\langle z_0 - x_0 , y_0 - x_0 \rangle \geq \|(z_0 - x_0 )\|^2 + \|(y_0 - x_0 )\|^2 \nonumber \\
&\|(z_0 - x_0 )\|^2 + \|(y_0 - x_0 )\|^2 - 2\langle z_0 - x_0 , y_0 - x_0 \rangle \leq 0 \nonumber \\
&\|(z_0 - x_0 ) - (y_0 - x_0 )\|^2 \leq 0 \nonumber \\
&\|z_0  - y_0\|^2 \leq 0 \nonumber
\end{align}
על פי הגדרת ה $\|\cdot \|$,
 $\|\cdot \| \geq 0 $
 ולכן הפתרון היחיד האפשרי לתוצאה שקיבלנו הוא ש : 
\begin{align}
&\|z_0  - y_0\|^2 = 0 \nonumber \\
&z_0  - y_0 = 0 \nonumber \\
&z_0 = y_0  \nonumber
\end{align}

כלומר הפתרונות שהנחנו שפותרים את בעיית האופטימיזציה זהים, ולכן קיים רק פתרון יחיד שפותר את בעיית האופטימיזציה 
\end{document}